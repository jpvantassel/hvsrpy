%% Generated by Sphinx.
\def\sphinxdocclass{report}
\documentclass[letterpaper,10pt,english,openany,oneside]{sphinxmanual}
\ifdefined\pdfpxdimen
   \let\sphinxpxdimen\pdfpxdimen\else\newdimen\sphinxpxdimen
\fi \sphinxpxdimen=.75bp\relax

\PassOptionsToPackage{warn}{textcomp}
\usepackage[utf8]{inputenc}
\ifdefined\DeclareUnicodeCharacter
% support both utf8 and utf8x syntaxes
  \ifdefined\DeclareUnicodeCharacterAsOptional
    \def\sphinxDUC#1{\DeclareUnicodeCharacter{"#1}}
  \else
    \let\sphinxDUC\DeclareUnicodeCharacter
  \fi
  \sphinxDUC{00A0}{\nobreakspace}
  \sphinxDUC{2500}{\sphinxunichar{2500}}
  \sphinxDUC{2502}{\sphinxunichar{2502}}
  \sphinxDUC{2514}{\sphinxunichar{2514}}
  \sphinxDUC{251C}{\sphinxunichar{251C}}
  \sphinxDUC{2572}{\textbackslash}
\fi
\usepackage{cmap}
\usepackage[T1]{fontenc}
\usepackage{amsmath,amssymb,amstext}
\usepackage{babel}



\usepackage{times}
\expandafter\ifx\csname T@LGR\endcsname\relax
\else
% LGR was declared as font encoding
  \substitutefont{LGR}{\rmdefault}{cmr}
  \substitutefont{LGR}{\sfdefault}{cmss}
  \substitutefont{LGR}{\ttdefault}{cmtt}
\fi
\expandafter\ifx\csname T@X2\endcsname\relax
  \expandafter\ifx\csname T@T2A\endcsname\relax
  \else
  % T2A was declared as font encoding
    \substitutefont{T2A}{\rmdefault}{cmr}
    \substitutefont{T2A}{\sfdefault}{cmss}
    \substitutefont{T2A}{\ttdefault}{cmtt}
  \fi
\else
% X2 was declared as font encoding
  \substitutefont{X2}{\rmdefault}{cmr}
  \substitutefont{X2}{\sfdefault}{cmss}
  \substitutefont{X2}{\ttdefault}{cmtt}
\fi


\usepackage[Bjarne]{fncychap}
\usepackage{sphinx}

\fvset{fontsize=\small}
\usepackage{geometry}

% Include hyperref last.
\usepackage{hyperref}
% Fix anchor placement for figures with captions.
\usepackage{hypcap}% it must be loaded after hyperref.
% Set up styles of URL: it should be placed after hyperref.
\urlstyle{same}
\addto\captionsenglish{\renewcommand{\contentsname}{Contents:}}

\usepackage{sphinxmessages}
\setcounter{tocdepth}{1}



\title{hvsrpy}
\date{Nov 12, 2019}
\release{0.0.1}
\author{Joseph P.\@{} Vantassel}
\newcommand{\sphinxlogo}{\vbox{}}
\renewcommand{\releasename}{Release}
\makeindex
\begin{document}

\pagestyle{empty}
\sphinxmaketitle
\pagestyle{plain}
\sphinxtableofcontents
\pagestyle{normal}
\phantomsection\label{\detokenize{index::doc}}



\chapter{Summary}
\label{\detokenize{index:summary}}
hvsrpy is a Python module for performing horizontal-to-vertical spectral ratio
processing. hvsrpy was developed by Joseph P. Vantassel with contributions
from Dana M. Brannon under the supervision of Professor Brady R. Cox at the
University of Texas at Austin. The fully-automated frequency-domain rejection
algorithm implemented in \sphinxtitleref{hvsrpy} was developed by Tianjian Cheng under the
supervision of Professor Brady R. Cox at the Univesity of Texas at Austin and
detailed in Cox et al. (in review).

The module includes two main class definitons \sphinxtitleref{Sensor3c} and \sphinxtitleref{Hvsr}. These
classes include various methods for creating and manipulating 3-component
sensor and horizontal-to-vertical spectral ratio objects.


\chapter{License Information}
\label{\detokenize{index:license-information}}\begin{quote}

Copyright (C) 2019 Joseph P. Vantassel (\sphinxhref{mailto:jvantassel@utexas.edu}{jvantassel@utexas.edu})

This program is free software: you can redistribute it and/or modify
it under the terms of the GNU General Public License as published by
the Free Software Foundation, either version 3 of the License, or
(at your option) any later version.

This program is distributed in the hope that it will be useful,
but WITHOUT ANY WARRANTY; without even the implied warranty of
MERCHANTABILITY or FITNESS FOR A PARTICULAR PURPOSE.  See the
GNU General Public License for more details.

You should have received a copy of the GNU General Public License
along with this program.  If not, see \textless{}https: //www.gnu.org/licenses/\textgreater{}.
\end{quote}


\chapter{Sensor3c Class}
\label{\detokenize{index:sensor3c-class}}\index{Sensor3c (class in hvsrpy)@\spxentry{Sensor3c}\spxextra{class in hvsrpy}}

\begin{fulllineitems}
\phantomsection\label{\detokenize{index:hvsrpy.Sensor3c}}\pysiglinewithargsret{\sphinxbfcode{\sphinxupquote{class }}\sphinxbfcode{\sphinxupquote{Sensor3c}}}{\emph{ns}, \emph{ew}, \emph{vt}}{}
Class for creating and manipulating 3-component sensor objects.
\begin{description}
\item[{Attributes:}] \leavevmode\begin{description}
\item[{ns, ew, vt}] \leavevmode{[}Timeseries{]}
TimeSeries object for each component.

\item[{ns\_f, ew\_f, vt\_f}] \leavevmode{[}FourierTransform{]}
FourierTransform object for each component.

\end{description}

\end{description}
\index{\_\_init\_\_() (Sensor3c method)@\spxentry{\_\_init\_\_()}\spxextra{Sensor3c method}}

\begin{fulllineitems}
\phantomsection\label{\detokenize{index:hvsrpy.Sensor3c.__init__}}\pysiglinewithargsret{\sphinxbfcode{\sphinxupquote{\_\_init\_\_}}}{\emph{ns}, \emph{ew}, \emph{vt}}{}
Initalize a 3-component sensor (Sensor3c) object.
\begin{description}
\item[{Args:}] \leavevmode\begin{description}
\item[{ns, ew, vt}] \leavevmode{[}timeseries{]}
Timeseries object for each component.

\end{description}

\item[{Returns:}] \leavevmode
Initialized 3-component sensor (Sensor3c) object.

\end{description}

\end{fulllineitems}

\index{bandpassfilter() (Sensor3c method)@\spxentry{bandpassfilter()}\spxextra{Sensor3c method}}

\begin{fulllineitems}
\phantomsection\label{\detokenize{index:hvsrpy.Sensor3c.bandpassfilter}}\pysiglinewithargsret{\sphinxbfcode{\sphinxupquote{bandpassfilter}}}{\emph{flow}, \emph{fhigh}, \emph{order}}{}
Bandpassfilter component TimeSeries.

Refer to \sphinxtitleref{SigProPy} documentation for details.

\end{fulllineitems}

\index{combine\_horizontals() (Sensor3c method)@\spxentry{combine\_horizontals()}\spxextra{Sensor3c method}}

\begin{fulllineitems}
\phantomsection\label{\detokenize{index:hvsrpy.Sensor3c.combine_horizontals}}\pysiglinewithargsret{\sphinxbfcode{\sphinxupquote{combine\_horizontals}}}{\emph{method='squared-average'}}{}
Combine two horizontal components (\sphinxtitleref{ns} and \sphinxtitleref{ew}).
\begin{description}
\item[{Args:}] \leavevmode\begin{description}
\item[{ratio\_type}] \leavevmode{[}\{‘squared-averge’, ‘geometric-mean’\}, optional{]}
Defines how the two horizontal components are combined 
to represent a single horizontal component. By default
the ‘squared-average’ approach is used.

\end{description}

\item[{Return:}] \leavevmode
A FourierTransform object representing the combined
horizontal component.

\end{description}

\end{fulllineitems}

\index{cosine\_taper() (Sensor3c method)@\spxentry{cosine\_taper()}\spxextra{Sensor3c method}}

\begin{fulllineitems}
\phantomsection\label{\detokenize{index:hvsrpy.Sensor3c.cosine_taper}}\pysiglinewithargsret{\sphinxbfcode{\sphinxupquote{cosine\_taper}}}{\emph{width}}{}
Cosine taper component TimeSeries.

Refer to \sphinxtitleref{SigProPy} documentation for details.

\end{fulllineitems}

\index{detrend() (Sensor3c method)@\spxentry{detrend()}\spxextra{Sensor3c method}}

\begin{fulllineitems}
\phantomsection\label{\detokenize{index:hvsrpy.Sensor3c.detrend}}\pysiglinewithargsret{\sphinxbfcode{\sphinxupquote{detrend}}}{}{}
Detrend component TimeSeries.

Refer to \sphinxtitleref{SigProPy} documentation for details.

\end{fulllineitems}

\index{from\_mseed() (Sensor3c class method)@\spxentry{from\_mseed()}\spxextra{Sensor3c class method}}

\begin{fulllineitems}
\phantomsection\label{\detokenize{index:hvsrpy.Sensor3c.from_mseed}}\pysiglinewithargsret{\sphinxbfcode{\sphinxupquote{classmethod }}\sphinxbfcode{\sphinxupquote{from\_mseed}}}{\emph{fname}}{}
Initialize a 3-component sensor (Sensor3c) object from a
.miniseed file.
\begin{description}
\item[{Args:}] \leavevmode\begin{description}
\item[{fname}] \leavevmode{[}str{]}
Name of miniseed file, full path may be used if desired.
The file should contain three traces with the 
appropriate channel names. Refer to the \sphinxtitleref{SEED} Manual 
\sphinxhref{https://www.fdsn.org/seed\_manual/SEEDManual\_V2.4.pdf}{here}.
for specifics.

\end{description}

\item[{Returns:}] \leavevmode
Initialized 3-component sensor (Sensor3c) object.

\end{description}

\end{fulllineitems}

\index{hv() (Sensor3c method)@\spxentry{hv()}\spxextra{Sensor3c method}}

\begin{fulllineitems}
\phantomsection\label{\detokenize{index:hvsrpy.Sensor3c.hv}}\pysiglinewithargsret{\sphinxbfcode{\sphinxupquote{hv}}}{\emph{windowlength}, \emph{bp\_filter}, \emph{taper\_width}, \emph{bandwidth}, \emph{resampling}, \emph{method}}{}
Prepare time series and fourier transforms then compute H/V.
\begin{description}
\item[{Args:}] \leavevmode\begin{description}
\item[{windowlength}] \leavevmode{[}float{]}
Length of time windows in seconds.

\item[{bp\_filter}] \leavevmode{[}dict{]}
Bandpass filter settings, of the form 
\{‘flag’:\sphinxtitleref{bool}, ‘flow’:\sphinxtitleref{float}, ‘fhigh’:\sphinxtitleref{float},
‘order’:\sphinxtitleref{int}\}.

\item[{taper\_width}] \leavevmode{[}float{]}
Width of cosine taper.

\item[{bandwidth}] \leavevmode{[}float{]}
Bandwidth of the Konno and Ohmachi smoothing window.

\item[{resampling}] \leavevmode{[}dict{]}
Resampling settings, of the form 
\{‘minf’:\sphinxtitleref{float}, ‘maxf’:\sphinxtitleref{float}, ‘nf’:\sphinxtitleref{int}, 
‘res\_type’:\sphinxtitleref{str}\}.

\item[{method}] \leavevmode{[}\{‘squared-averge’, ‘geometric-mean’\}{]}
Refer to method \sphinxtitleref{combine\_horizontals} for details.

\end{description}

\item[{Returns:}] \leavevmode
Initialized Hvsr object.

\item[{Notes:}] \leavevmode
More information for the above arguements can be found in
the documenation of \sphinxtitleref{SigProPy}.

\end{description}

\end{fulllineitems}

\index{resample() (Sensor3c method)@\spxentry{resample()}\spxextra{Sensor3c method}}

\begin{fulllineitems}
\phantomsection\label{\detokenize{index:hvsrpy.Sensor3c.resample}}\pysiglinewithargsret{\sphinxbfcode{\sphinxupquote{resample}}}{\emph{fmin}, \emph{fmax}, \emph{fn}, \emph{res\_type}, \emph{inplace}}{}
Resample component FourierTransforms.

Refer to \sphinxtitleref{SigProPy} documentation for details.

\end{fulllineitems}

\index{smooth() (Sensor3c method)@\spxentry{smooth()}\spxextra{Sensor3c method}}

\begin{fulllineitems}
\phantomsection\label{\detokenize{index:hvsrpy.Sensor3c.smooth}}\pysiglinewithargsret{\sphinxbfcode{\sphinxupquote{smooth}}}{\emph{bandwidth}}{}
Smooth component FourierTransforms.

Refer to \sphinxtitleref{SigProPy} documentation for details.

\end{fulllineitems}

\index{split() (Sensor3c method)@\spxentry{split()}\spxextra{Sensor3c method}}

\begin{fulllineitems}
\phantomsection\label{\detokenize{index:hvsrpy.Sensor3c.split}}\pysiglinewithargsret{\sphinxbfcode{\sphinxupquote{split}}}{\emph{windowlength}}{}
Split component TimeSeries.

Refer to \sphinxtitleref{SigProPy} documentation for details.

\end{fulllineitems}

\index{transform() (Sensor3c method)@\spxentry{transform()}\spxextra{Sensor3c method}}

\begin{fulllineitems}
\phantomsection\label{\detokenize{index:hvsrpy.Sensor3c.transform}}\pysiglinewithargsret{\sphinxbfcode{\sphinxupquote{transform}}}{}{}
Perform Fourier transform on components.
\begin{description}
\item[{Returns:}] \leavevmode
\sphinxtitleref{None}, redefines attributes \sphinxtitleref{ew\_f}, \sphinxtitleref{ns\_f}, and \sphinxtitleref{vt\_f} as 
FourierTransform objects for each component.

\end{description}

\end{fulllineitems}


\end{fulllineitems}



\chapter{Hvsr Class}
\label{\detokenize{index:hvsr-class}}\index{Hvsr (class in hvsrpy)@\spxentry{Hvsr}\spxextra{class in hvsrpy}}

\begin{fulllineitems}
\phantomsection\label{\detokenize{index:hvsrpy.Hvsr}}\pysiglinewithargsret{\sphinxbfcode{\sphinxupquote{class }}\sphinxbfcode{\sphinxupquote{Hvsr}}}{\emph{amplitude}, \emph{frequency}, \emph{find\_peaks=True}}{}
Class for creating and manipulating horizontal-to-vertical
spectral ratio objects.
\begin{description}
\item[{Attributes:}] \leavevmode\begin{description}
\item[{amp}] \leavevmode{[}ndarray{]}
Array of H/V amplitudes. Each row represents an individual
curve and each column a frequency.

\item[{frq}] \leavevmode{[}ndarray{]}
Vector of frequencies, corresponding to each column.

\item[{n\_windows}] \leavevmode{[}int{]}
Number of windows in Hvsr object.

\item[{valid\_window\_indices}] \leavevmode{[}ndarray{]}
Array of indices indicating valid windows.

\end{description}

\end{description}
\index{\_\_init\_\_() (Hvsr method)@\spxentry{\_\_init\_\_()}\spxextra{Hvsr method}}

\begin{fulllineitems}
\phantomsection\label{\detokenize{index:hvsrpy.Hvsr.__init__}}\pysiglinewithargsret{\sphinxbfcode{\sphinxupquote{\_\_init\_\_}}}{\emph{amplitude}, \emph{frequency}, \emph{find\_peaks=True}}{}
Initialize a Hvsr oject from an amplitude and frequency
vector.
\begin{description}
\item[{Args:}] \leavevmode\begin{description}
\item[{amplitude}] \leavevmode{[}ndarray{]}
Array of H/V amplitudes. Each row represents an individual
curve and each column a frequency.

\item[{frequency}] \leavevmode{[}ndarray{]}
Vector of frequencies, corresponding to each column.

\end{description}

\item[{Returns:}] \leavevmode
Initialized Hvsr object.

\end{description}

\end{fulllineitems}

\index{find\_peaks() (Hvsr static method)@\spxentry{find\_peaks()}\spxextra{Hvsr static method}}

\begin{fulllineitems}
\phantomsection\label{\detokenize{index:hvsrpy.Hvsr.find_peaks}}\pysiglinewithargsret{\sphinxbfcode{\sphinxupquote{static }}\sphinxbfcode{\sphinxupquote{find\_peaks}}}{\emph{amp}, \emph{**kwargs}}{}
Returns the indices of all peaks in \sphinxtitleref{amp}.

Wrapper method for scipy.signal.find\_peaks function.
\begin{description}
\item[{Args:}] \leavevmode\begin{description}
\item[{amp}] \leavevmode{[}ndarray{]}
Vector or array of amplitudes. See \sphinxtitleref{amp} attribute for 
details.

\item[{{\color{red}\bfseries{}**}kwargs}] \leavevmode{[}dict{]}
Refer to \sphinxtitleref{scipy.signal.find\_peaks} documentation
\sphinxhref{https://docs.scipy.org/doc/scipy/reference/generated/scipy.signal.find\_peaks.html}{here}.

\end{description}

\item[{Returns:}] \leavevmode\begin{description}
\item[{peaks}] \leavevmode{[}ndarray or list{]}
\sphinxtitleref{ndarray} or \sphinxtitleref{list} of \sphinxtitleref{ndarrays} (one per window) of
peak indices.

\item[{properties}] \leavevmode{[}dict{]}
Refer to \sphinxtitleref{scipy.signal.find\_peaks} documentation.

\end{description}

\end{description}

\end{fulllineitems}

\index{mc\_peak() (Hvsr method)@\spxentry{mc\_peak()}\spxextra{Hvsr method}}

\begin{fulllineitems}
\phantomsection\label{\detokenize{index:hvsrpy.Hvsr.mc_peak}}\pysiglinewithargsret{\sphinxbfcode{\sphinxupquote{mc\_peak}}}{\emph{distribution='log-normal'}}{}
Peak of mean H/V curve.
\begin{description}
\item[{Args:}] \leavevmode\begin{description}
\item[{distribution}] \leavevmode{[}\{‘normal’, ‘log-normal’\}, optional{]}
Refer to method \sphinxtitleref{mean\_curve} for details.

\end{description}

\item[{Returns:}] \leavevmode
Frequency associated with the peak of the mean H/V curve.

\end{description}

\end{fulllineitems}

\index{mean\_curve() (Hvsr method)@\spxentry{mean\_curve()}\spxextra{Hvsr method}}

\begin{fulllineitems}
\phantomsection\label{\detokenize{index:hvsrpy.Hvsr.mean_curve}}\pysiglinewithargsret{\sphinxbfcode{\sphinxupquote{mean\_curve}}}{\emph{distribution='log-normal'}}{}
Return mean H/V curve.
\begin{description}
\item[{Args:}] \leavevmode\begin{description}
\item[{distribution}] \leavevmode{[}\{‘normal’, ‘log-normal’\}, optional{]}
Assumed distribution of mean curve, default is 
\sphinxtitleref{log-normal}.

\end{description}

\item[{Returns:}] \leavevmode
Mean H/V curve as \sphinxtitleref{ndarray} according to the distribution
specified.

\item[{Raises:}] \leavevmode\begin{description}
\item[{KeyError:}] \leavevmode
If \sphinxtitleref{distribution} does not match the available options.

\end{description}

\end{description}

\end{fulllineitems}

\index{mean\_f0() (Hvsr method)@\spxentry{mean\_f0()}\spxextra{Hvsr method}}

\begin{fulllineitems}
\phantomsection\label{\detokenize{index:hvsrpy.Hvsr.mean_f0}}\pysiglinewithargsret{\sphinxbfcode{\sphinxupquote{mean\_f0}}}{\emph{distribution='log-normal'}}{}
Return mean value of \sphinxtitleref{f0} of valid timewindows.
\begin{description}
\item[{Args:}] \leavevmode\begin{description}
\item[{distribution}] \leavevmode{[}\{‘normal’, ‘log-normal’\}{]}
Assumed distribution of \sphinxtitleref{f0}, default is \sphinxtitleref{log-normal}.

\end{description}

\item[{Returns:}] \leavevmode
Mean value of \sphinxtitleref{f0} according to the distribution specified.

\item[{Raises:}] \leavevmode\begin{description}
\item[{KeyError:}] \leavevmode
If \sphinxtitleref{distribution} does not match the available options.

\end{description}

\end{description}

\end{fulllineitems}

\index{nstd\_curve() (Hvsr method)@\spxentry{nstd\_curve()}\spxextra{Hvsr method}}

\begin{fulllineitems}
\phantomsection\label{\detokenize{index:hvsrpy.Hvsr.nstd_curve}}\pysiglinewithargsret{\sphinxbfcode{\sphinxupquote{nstd\_curve}}}{\emph{n}, \emph{distribution}}{}
Return nth standard deviation curve.
\begin{description}
\item[{Args:}] \leavevmode\begin{description}
\item[{n}] \leavevmode{[}float{]}
Number of standard deviations away from the mean curve.

\item[{distribution}] \leavevmode{[}\{‘log-normal’, ‘normal’\}, optional{]}
Assumed distribution of mean curve, the default is
‘log-normal’.

\end{description}

\item[{Return:}] \leavevmode
nth standard deviation curve as an \sphinxtitleref{ndarray}.

\end{description}

\end{fulllineitems}

\index{nstd\_f0() (Hvsr method)@\spxentry{nstd\_f0()}\spxextra{Hvsr method}}

\begin{fulllineitems}
\phantomsection\label{\detokenize{index:hvsrpy.Hvsr.nstd_f0}}\pysiglinewithargsret{\sphinxbfcode{\sphinxupquote{nstd\_f0}}}{\emph{n}, \emph{distribution}}{}
Return nth standard deviation of \sphinxtitleref{f0}.
\begin{description}
\item[{Args:}] \leavevmode\begin{description}
\item[{n}] \leavevmode{[}float{]}
Number of standard deviations away from the mean \sphinxtitleref{f0}.

\item[{distribution}] \leavevmode{[}\{‘log-normal’, ‘normal’\}, optional{]}
Assumed distribution of \sphinxtitleref{f0}, the default is
‘log-normal’.

\end{description}

\item[{Return:}] \leavevmode
nth standard deviation of \sphinxtitleref{f0} as \sphinxtitleref{float}.

\end{description}

\end{fulllineitems}

\index{peak\_amp() (Hvsr property)@\spxentry{peak\_amp()}\spxextra{Hvsr property}}

\begin{fulllineitems}
\phantomsection\label{\detokenize{index:hvsrpy.Hvsr.peak_amp}}\pysigline{\sphinxbfcode{\sphinxupquote{property }}\sphinxbfcode{\sphinxupquote{peak\_amp}}}
Return valid peaks amplitude vector.

\end{fulllineitems}

\index{peak\_frq() (Hvsr property)@\spxentry{peak\_frq()}\spxextra{Hvsr property}}

\begin{fulllineitems}
\phantomsection\label{\detokenize{index:hvsrpy.Hvsr.peak_frq}}\pysigline{\sphinxbfcode{\sphinxupquote{property }}\sphinxbfcode{\sphinxupquote{peak\_frq}}}
Return valid peaks frequency vector.

\end{fulllineitems}

\index{reject\_windows() (Hvsr method)@\spxentry{reject\_windows()}\spxextra{Hvsr method}}

\begin{fulllineitems}
\phantomsection\label{\detokenize{index:hvsrpy.Hvsr.reject_windows}}\pysiglinewithargsret{\sphinxbfcode{\sphinxupquote{reject\_windows}}}{\emph{n=2}, \emph{max\_iterations=50}, \emph{distribution\_f0='log-normal'}, \emph{distribution\_mc='log-normal'}}{}
Perform rejection of H/V windows using the method proposed by
Cox et al. (in review).
\begin{description}
\item[{Args:}] \leavevmode\begin{description}
\item[{n}] \leavevmode{[}float, optional{]}
Number of standard deviations from the mean (default
value is 2).

\item[{max\_iterations}] \leavevmode{[}int, optional{]}
Maximum number of rejection iterations (default value is
50).

\item[{distribution\_f0}] \leavevmode{[}\{‘log-normal’, ‘normal’\}, optional{]}
Assumed distribution of \sphinxtitleref{f0} from time windows, the
default is ‘log-normal’.

\item[{distribution\_mc}] \leavevmode{[}\{‘log-normal’, ‘normal’\}, optional{]}
Assumed distribution of mean curve, the default is
‘log-normal’.

\end{description}

\item[{Returns:}] \leavevmode\begin{description}
\item[{c\_iteration}] \leavevmode{[}int{]}
Number of iterations required for convergence.

\end{description}

\end{description}

\end{fulllineitems}

\index{std\_curve() (Hvsr method)@\spxentry{std\_curve()}\spxextra{Hvsr method}}

\begin{fulllineitems}
\phantomsection\label{\detokenize{index:hvsrpy.Hvsr.std_curve}}\pysiglinewithargsret{\sphinxbfcode{\sphinxupquote{std\_curve}}}{\emph{distribution='log-normal'}}{}
Sample standard deviation associate with the mean H/V curve.
\begin{description}
\item[{Args:}] \leavevmode\begin{description}
\item[{distribution}] \leavevmode{[}\{‘normal’, ‘log-normal’\}, optional{]}
Assumed distribution of H/V curve, default is
\sphinxtitleref{log-normal}.

\end{description}

\item[{Returns:}] \leavevmode
Sample standard deviation of H/V curve as \sphinxtitleref{ndarray}
according to the distribution specified.

\item[{Raises:}] \leavevmode\begin{description}
\item[{ValueError:}] \leavevmode
If only single time window is defined.

\item[{KeyError:}] \leavevmode
If \sphinxtitleref{distribution} does not match the available options.

\end{description}

\end{description}

\end{fulllineitems}

\index{std\_f0() (Hvsr method)@\spxentry{std\_f0()}\spxextra{Hvsr method}}

\begin{fulllineitems}
\phantomsection\label{\detokenize{index:hvsrpy.Hvsr.std_f0}}\pysiglinewithargsret{\sphinxbfcode{\sphinxupquote{std\_f0}}}{\emph{distribution='log-normal'}}{}
Return sample standard deviation of \sphinxtitleref{f0} of valid
timewindows.
\begin{description}
\item[{Args:}] \leavevmode\begin{description}
\item[{distribution}] \leavevmode{[}\{‘normal’, ‘log-normal’\}, optional{]}
Assumed distribution of \sphinxtitleref{f0}, default is \sphinxtitleref{log-normal}.

\end{description}

\item[{Returns:}] \leavevmode\begin{description}
\item[{std}] \leavevmode{[}float{]}
Sample standard deviation value according to the
distribution specified.

\end{description}

\item[{Raises:}] \leavevmode\begin{description}
\item[{KeyError:}] \leavevmode
If \sphinxtitleref{distribution} does not match the available options.

\end{description}

\end{description}

\end{fulllineitems}

\index{to\_file\_like\_geopsy() (Hvsr method)@\spxentry{to\_file\_like\_geopsy()}\spxextra{Hvsr method}}

\begin{fulllineitems}
\phantomsection\label{\detokenize{index:hvsrpy.Hvsr.to_file_like_geopsy}}\pysiglinewithargsret{\sphinxbfcode{\sphinxupquote{to\_file\_like\_geopsy}}}{\emph{fname}, \emph{distribution\_f0}, \emph{distribution\_mc}}{}
Save H/V data to file following the Geopsy format.
\begin{description}
\item[{Args:}] \leavevmode\begin{description}
\item[{fname}] \leavevmode{[}str{]}
Name of file to save the results, may be a full or
relative path.

\item[{distribution\_f0}] \leavevmode{[}\{‘log-normal’, ‘normal’\}, optional{]}
Assumed distribution of \sphinxtitleref{f0} from the time windows, the
default is ‘log-normal’.

\item[{distribution\_mc}] \leavevmode{[}\{‘log-normal’, ‘normal’\}, optional{]}
Assumed distribution of mean curve, the default is
‘log-normal’.

\end{description}

\item[{Returns:}] \leavevmode
\sphinxtitleref{None}, writes file to disk.

\end{description}

\end{fulllineitems}

\index{update\_peaks() (Hvsr method)@\spxentry{update\_peaks()}\spxextra{Hvsr method}}

\begin{fulllineitems}
\phantomsection\label{\detokenize{index:hvsrpy.Hvsr.update_peaks}}\pysiglinewithargsret{\sphinxbfcode{\sphinxupquote{update\_peaks}}}{\emph{**kwargs}}{}
Update \sphinxtitleref{peaks} attribute with the lowest frequency, highest
amplitude peak.
\begin{description}
\item[{Args:}] \leavevmode\begin{description}
\item[{{\color{red}\bfseries{}**}kwargs:}] \leavevmode
Refer to static method \sphinxtitleref{find\_peaks} documentation.

\end{description}

\item[{Returns:}] \leavevmode
\sphinxtitleref{None}, update \sphinxtitleref{peaks} attribute.

\end{description}

\end{fulllineitems}


\end{fulllineitems}




\renewcommand{\indexname}{Index}
\printindex
\end{document}